\section*{Eingangsimpuls erfassen }
Schreiben Sie ein Assemblerprogramm für den 16C83, das einen Eingangsimpuls (0,1 ms bis 0,5
ms) erfasst und daraus einen 8 x so langen Ausgangsimpuls erzeugt. Der Ausgangsimpuls soll erst
dann erscheinen, wenn der Eingangsimpuls wieder weg ist. Die Quarzfrequenz beträgt 4 MHz; ein
einfacher Befehl benoetigt somit 1 $\mu s$.\\
\texttt{Pegel prüfen bis eine $1$ kommt}
\begin{lstlisting}[language=avr]
Label1
  BTFSC   porta, 0
  GOTO    Label1      ;warten auf low
Label2
  BTFSS   porta, 0
  GOTO    Label2      ;wenn das ueberspringt gab es eine ssteigende Flanke
                      ;Ab jetzt messen -> solange incrementieren
\end{lstlisting}
\texttt{8-bitzähler geht bis 512 $\mu s$}
\begin{lstlisting}[language=avr]
Label3
  INCF    Dauer, 1    ;1 Takt
  BTFSC   porta, 0    ;1
  GOTO    Label3      ;2 Vier Takte fuer den letzten 3
   
;Ausgabe
Label3
  MOVLW   8           ;ein Register wird auf 8 gesetzt
  MOVWF   Schleife    ;ist eine Variable
  BSF     porta, 1   
Label5
  MOVF    Dauer, W
  MOVWF   Counter
Label4
  DECF    Counter
  BTFSS   status,zero
  GOTO    Label4
  DECTSZ  Schleife
  GOTO    Label5
  BCF     porta, 1
  GOTO    Label2
\end{lstlisting}