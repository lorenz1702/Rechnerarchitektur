\subsubsection*{Testprog1}
\begin{lstlisting}[language=avr]
  ;*******************************************************
  ; Testprog1.src
  ;*******************************************************
    device 16f84
  ;Symbol definieren
  status    equ 3
  zero      equ 2
  rp0 	    equ 5
  trisa	    equ 5
  trisb	    equ 6
  porta	    equ 5
  portb	    equ 6

  ;Hex-Zahlen: h am Ende, bei Zahlen mit Buchstaben an erster Stelle eine 0 davor
  wert	    equ 0ch
  alterw	  equ 13
  counter   equ 14
  
  
  org       0
    
  ;Einsprung beim Einschalten (Power on)
  cold
    bsf	    status,rp0	;auf Bank 1 umschalten
    movlw	  0
    movwf	  trisb		    ;PortB wird komplett als Ausgang geschaltet
    bcf 	  trisa,3		  ;RA3 wird Ausgang (Carry)
    bcf	    status,rp0	;zurueck auf Bank 0
  
  ;Definieren von alterw mit aktuellem Wert an RA0
    movf	  porta,w		  ;PortA lesen
    andlw	  00000001b
    movwf	  alterw

  ;Hauptschleife
  loop
    clrf	  counter		  ;Reset und Startwert
    clrf	  portb

  loop1
  ;Reset aktiv?
    btfss	  porta,1		  ;Reseteingang
    goto 	  loop
  ;Inhibit aktiv?
    btfsc	  porta,2		  ;Inhibiteingang
    goto 	  loop1
  
  ;Takteingang lesen
    movf	  porta,w		  ;PortA komplett eingelesen
    andlw	  1		        ;Nur R0 ist von Interesse
  
    xorwf	  alterw,w	  ;Wenn beide gleich, keine Flanke
    btfsc	  status,zero	;Beide gleich, Zero gesetzt
    goto	  loop1		    ;Nichts passiert
    movlw   1
    xorwf	  alterw		  ;Beinhaltet neuen Pegel an RA0
    btfss	  alterw,0
    goto	  loop1
  
  ;Richtige Flanke gefunden
    bcf	    porta,3
    incf	  counter		  ;Zaehler erhoehen
    movf 	  counter,w
    movwf	  portb
    btfss	  status,zero	;Zaehlerueberlauf
    goto	  loop1
    bsf	    porta,3		  ;Carryausgang setzen
    goto	  loop1
  
    end

\end{lstlisting}